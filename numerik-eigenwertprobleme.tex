\documentclass[%
a4paper,
%empty,		% keine Seitenzahlen
%a5paper,	% alle weiteren Papierformat einstellbar
11pt,		% Schriftgröße (12pt, 11pt (Standard))
leqno,		% Nummerierung von Gleichungen links
%fleqn,		% Ausgabe von Gleichungen linksbündig
]
{scrartcl}

%% Deutsche Anpassungen 
\usepackage[ngerman]{babel}
\usepackage[T1]{fontenc}
\usepackage[utf8]{inputenc}

%obligatorischer Mathekram:
\usepackage{amssymb,amstext,dsfont,trsym,pifont}
\usepackage[sumlimits]{amsmath}
\usepackage{eulervm}

%nützliche Extras:
\usepackage{array,
hhline,
longtable,
tabularx,
enumerate,
hyperref,
color,
setspace,
booktabs,
cite,
caption,
lineno,
lastpage,
algorithm,
ulem,
}

\usepackage[amsmath,thmmarks]{ntheorem}
% meine Theoremdefinitionen:
% Definitionen:
\theoremstyle{plain}
\theoremheaderfont{\bfseries}
\theorembodyfont{}
\theoremseparator{\ }
%\theoremprework{\hfill \rule{0.5\textwidth}{1pt} \hspace*{0.25\textwidth} }
%\theorempostwork{\rule}
%\theoremindent2ex
\newtheorem{mydef}{Definition}[section]

%Sätze:
\theoremstyle{plain}
\theoremheaderfont{\bfseries}
\theorembodyfont{}
\theoremsymbol{$\Box$}
\newtheorem{mysatz}[mydef]{Satz}

% Randverwaltung (entweder geometry oder =fullpage=)
%\usepackage[left=1cm,right=1cm,top=1cm,bottom=1cm,includeheadfoot]{geometry}
\usepackage[cm,
%headings,
]{fullpage}

% die fancy-Header:
\usepackage{fancyhdr}
\fancyhf{}

%\fancyhead[L]{}
%\fancyhead[C]{}
%\fancyhead[R]{\nouppercase \leftmark}
\fancyfoot[L]{}
\fancyfoot[C]{}
\fancyfoot[RO,LE]{\thepage}
%Linie oben/unten
\renewcommand{\headrulewidth}{0.0pt}
\renewcommand{\footrulewidth}{0.0pt}

%kein Einrücken der Paragraphen
\parindent 0pt

%% Packages für Grafiken & Abbildungen
%\usepackage{graphicx}
%\usepackage{subfig}    %%Teilabbildungen in einer Abbildung
%\usepackage{tikz}      %%TeX ist kein Zeichenprogramm
%\usepackage[all]{xy}
%\usepackage{pst-all}

\pagestyle{fancy}

\begin{document}

\setcounter{section}{6}

\section{Eigenwertprobleme}
\label{sec:Eigenwertprobleme}

\textbf{Motivation:} Mechanische Eigenschwingungen, z.B. Saite oder Membran.

Bewegungsgleichung ist die Wellengleichung mit der Auslenkung $u(x,t)$ an der Stelle $x$ zum Zeitpunkt $t$.\\

1-D Saite:
\[
\frac{\partial^2 u}{\partial t^2} = C \cdot \frac{\partial^2 u}{\partial x^2}
\]
2-D Membran:
\[
\frac{\partial^2 u}{\partial t^2} = C \triangle u - C \left( \frac{\partial^2 u}{\partial x^2} + \frac{\partial^2 u}{\partial y^2} \right)
\]
jeweils mit Randberechnung wie $\left. u \right|_{\text{Rand}}=0$.\\

\textbf{Diskretisierung:}
\begin{align*}
  y''(t) & = -Ay(t) \qquad\qquad  A \in \mathbb{R}^{n \times n} \text{ spd., } y(t) \in \mathbb{R}^n\\
  \intertext{\textbf{stationäre Lösung:}}
  \text{Ansatz: } y(t) & = e^{i \omega t} y \qquad \qquad y \in \mathbb{R}^n, \omega \text{ Eigenfrequenz}\\
  \Rightarrow e^{i \omega t} (i \omega)^2 y & = -A e^{i \omega t} y\\
  \Rightarrow Ay & = \omega^2 y = \lambda y \text{ mit } \lambda = \omega^2
\end{align*}

\subsection{Grundlagen} % (fold)
\label{sub:Grundlagen}

\begin{mydef}
Sei $A \in \mathbb{C}^{n \times n}, x \in \mathbb{C}^n \backslash \left\{ 0 \right\}, \lambda \in \mathbb{C}$ und $Ax = \lambda x$. Dann heißt $x$ ein (Rechts-)\textbf{Eigenvektor} (EV) von $A$ zum \textbf{Eigenwert} (EW) $\lambda$ von $A$. Das Paar $(\lambda, x)$ heißt ein Eigenpaar von $A$. Die Menge aller Eigenwerte von $A$
\[
\sigma(A) := \left\{ \lambda \ | \ \lambda \text{ ist Eigenwert von } A \right\}
\]
heißt das \textbf{Spektrum} von $A$.
\end{mydef}
\textit{Bemerkung:}\\[-1.3cm] %rrrrreal dirrty (but who cares anyway when it looks good?!)
\begin{align*}
Ax = \lambda x & \Leftrightarrow (\lambda I - A)x = 0 \\
 & \Leftrightarrow \det(\lambda I - A)=0\\
 & \Leftrightarrow \lambda \text{ ist Nullstelle des charakteristischen Polynoms.}
\end{align*}
Ist $x$ ein Eigenvektor, so auch $\theta x$ mit $\theta \neq 0$. Deshalb werden Eigenvektoren oft normiert, d.h. $\| x \|_2=1$.\\

\begin{mysatz}
Das charakteristische Polynom $p_A(\lambda) = \det(\lambda I - A)$ von $A$ besitzt die eindeutige Faktorisierung
\[
p_A(\lambda) = (\lambda - \lambda_1)^{m_1} \cdot (\lambda - \lambda_2)^{m_2} \cdot \ldots \cdot (\lambda - \lambda_k)^{n_k}
\]
mit den $k$ paarweise verschiedenen Eigenwerten $\lambda_k$ von $A$ mit $m_1 + \ldots + m_k = n$.

Dabei ist $m_i$ die algebraische Vielfachheit von $\lambda_i$.\newline

\textbf{Beweis:} Fundamentalsatz der Algebra.
\end{mysatz}

% subsection Grundlagen (end)

% section Eigenwertprobleme (end)

\end{document}
